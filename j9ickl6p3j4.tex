\section{Results and discussion}
\subsection{Characteristics of swine wastewater samples}
The characteristics of SW samples from different production stages in the studied farm is presented in Table 3. Regarding the different production stages, pH was 7.83 with some variability; one way ANOVA analysis showed significant differences among production stages with higher pH values for weaning and fattening phases (8.15 and 8.05, respectively). Also significant was the variation in EC values with a maximum of 8635 mS.cm\textsuperscript{-1} in gestating sows probably due to the presence of salts in diets \cite{Mart_nez_Suller_2008,Moral_2005,Moral_2008}. In terms of ORP, although no significant differences were obtained among the production stages, more negative ORP values were obtained in farrowing and fattening wastewaters (-55.5 and -48.0, respectively) likely because of higher particulate and dissolves organics matter loads which often have a negative surface charge \cite{Hjorth_2011}.\\
In the present work, the ANOVA test has shown significant differences (P \textminus 0.05) in terms of organic content as measured by COD and BOD\textsubscript{5} for the different stages considered. The COD mean values were only 283 mg.L\textsuperscript{-1} for farrowing, being notably higher for the fattenign phase (9200 mg.L\textsuperscript{-1}). This difference is probably due to different feed strategies (diets) between production stages. In terms of BOD, Tukey’s HSD test revealed no significant differences between gestation, weaners, and fattening; only farrowing wastewaters presented significantly low biodegradable organic content (153 mg.L\textsuperscript{-1}).\\
With respect to solids, ANOVA showed significant differences between the farming stages for every solids parameter. The highest and lowest TS (12088-752 mg.L\textsuperscript{-1}), TVS (5220-398 mg.L\textsuperscript{-1}) and TDS (6685-405mg.L\textsuperscript{-1}) contents were found in fattening and farrowing, respectively, with the difference likely explained by different nutrient digestibility at different growth stages \cite{Zhang_2014}. Furthermore, Tukey’s HSD test revealed the existence of no significantly differences between gestation and weaning phased in all parameters.\\
Concerning elemental composition, fattening wastewaters exhibited higher C, H and N contents than the other types of wastewaters. There was no significant difference in elemenal composition between the remaining production phases as revealed by Tukey's HSD test. The average S content of the different wastewater samples was 1.06\%. Finally, carbon- to-nitrogen ratio (C/N) is important for the activity of anaerobic microorganisms. Past studies indicate that the C/N ratio should be between 20:1 and 30:1 in order to achieve desirable conditions for the digestion process \cite{1,Fricke_2007}. The C/N ratios found for weaning and fattening phases (41 and 15, respectively) may be unsuitable for optimal digestion of these wastewaters.