\section{Results and discussion}
\subsection{Cha