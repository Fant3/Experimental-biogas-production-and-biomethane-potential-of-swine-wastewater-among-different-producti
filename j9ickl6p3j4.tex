\section{Results and discussion}
\subsection{Characteristics of swine wastewater samples}
The characteristics of SW samples from different production stages in the studied farm is presented in Table 3. Regarding the different production stages, pH was 7.83 with some variability; one way ANOVA analysis showed significant differences among production stages with higher pH values for weaning and fattening phases (8.15 and 8.05, respectively). Also significant was the variation in EC values with a maximum of 8635 in gestating sows probably due to the presence of salts in diet. In terms of ORP, although no significant differences were obtained among the production stages, more negative ORP values were obtained in farrowing and fattening wastewaters (-55.5 and -48.0, respectively) likely because of higher organics loads which often have a negative surface charge.
In the present work, the ANOVA test has shown significant differences (P \textminus 0.05) in terms of organic content as measured by COD and BOD\textsubscript{5} for the different stages considered. The COD mean values were only 283 mg.L\textsuperscript{-1} for farrowing, being notably higher for the fattenign phase (9200 mg.L\textsuperscript{-1}).  