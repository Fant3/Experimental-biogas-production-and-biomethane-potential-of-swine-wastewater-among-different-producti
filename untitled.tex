\section{Introduction and scope}
\subsection{General considerations about swine wastewater (SW) production}

Livestock production, including swine production, is one of the most important agricultural sub-sectors in the European Union (EU), with a long tradition across all member states. In 2015, the pig farming sector accounted for more than 8\% of agricultural output \cite{Eurostat2016} in the EU with most of the production depending on intensive methods conducted in large farms. In fact, the current trend across Europe involves large numbers of animals concentrated in small regional clusters in countries like Denmark, Germany, Spain, France, Poland and the Netherlands, which accounted for more than 70 \% of pig population \cite{Eurostat2016} in 2015. The role of Portugal in this European pig farming scenario is still very limited with the country hosting a pig population of about 2 million heads in a universe of 148 million (about 1,5 \%, 2015 data) \cite{Eurostat2016}; however, the general trend for intensive production, concentration in small areas, and increase on the average farm size is the same as in Europe, with many environmental impacts at a local level well documented.\\
Despite meat consumption being somewhat constant in last few years within the EU \cite{OECD2016}, uncertainties surrounding increased competition and regulation still remain. This is the case not only for Portuguese farm owners, but to the whole pig farming sector at EU level due to criticisms related to the environmental impact of farm activities and challenges related to energy use and economical viability at farm level. Therefore, the need for the development of novel manure management and processing techniques is present and has drawn extensive attention and interest in recent years among scientists and private farm owners. In fact, swine wastes like the liquid fraction of swine manure, are increasingly viewed as an important bioenergy feedstock with an associated economic value. Beyond its traditional application as soil fertilizer, swine wastewater is believed to be a promising option for biogas and hydrogen production with the capability to at least complement the energy demand of many farms while contributing to alleviate the financial pressure on farm owners. This scenario thus opens a new window for research ventures related to the remediation of swine-derived wastewaters with the aim of producing energy and value-added products while respecting environmental regulations.

Swine wastes are a complex biogenic organic-inorganic waste generated by the natural process of animal food digestion and can be defined as a mixture of animal excrements (45\% solid and 55\% liquid \cite{Villamar_2011}) and process waters used for sanitary purposes in swine farms \cite{Vassilev_2010, Marszaek2014}. The general classification of these byproduct varieties in pig breeding can be preliminary divided in three groups according to the dry matter content of the wastes, namely: wastewaters (less than 5\% DM), slurries (between 5-15\% DM) and solid manures (more than 15\% DM). As the main by-product of swine farms, the liquid fraction of swine manure is typically generated in large quantities in non-bedding intensive animal farming being the subject of great environmental concern in the modern world as a major cause of point-source pollution from wastewater discharge or overfertilization of soils. Specifically, high amounts of discharged nutrients and organic matter can lead to eutrophication of water bodies, soil and air pollution from accumulation of nutrients, gas emissions such as ammonia, as well as odors. Moreover, waste streams derived from pig production are also a source of microbiological and heavy metal contamination, which can lead to human health hazards. For example, livestock excrete many different pathogenic microorganisms of relevance to human health which can be water-borne or food-borne and enter human food chain through agricultural contamination in many irrigation systems \cite{Andersson2013}. Within this framework, European policy makers have advanced with a series of regulations and instruments in order to minimize environmental concerns. In 1991, the Nitrate Directive \cite{CounciloftheEuropeanCommunities1991} for the protection of water bodies was adopted and established a limit of 50 mg.L\textsuperscript{-1} for wastewater discharge. Later, the Water Framework Directive \cite{EuropeanParliament2000} imposed standards for "good quality" of surface and underground waters. As a result, legislation was adopted in Portugal regarding pollutant concentrations for treated wastewaters. The maximum permissible values related with industrial wastewaters discharge in Portugal\, including swine farm effluents, are summarized in Table \ref{table1}. Also, the Gothenburg Protocol \cite{EuropeanParliament2001} was adopted in 1999 setting national targets for five pollutants (SO\textsubscript{2}, NO\textsubscript{x}, volatile organic compounds (VOC), NH\textsubscript{3} and fine particulate matter) responsible for acidification, eutrophication and ground-level ozone pollution. As a result of this effort,SO\textsubscript{2} emissions were reduced by 82\%, NO\textsubscript{x} emissions by 47\%, non-methane VOC emissions by 56\% and NH\textsubscript{3} emissions by 28\% in the EU between 1990 and 2010. More recently, new emission commitments were adopted for 2020 and 2030 \cite{EuropeanParliament2016}. Of special interest for swine production activities are NH\textsubscript{3} emissions with Portugal aiming a reduction target of 7\% and 15\% for 2020 and 2030 relative to 2005 levels.

At present, the main challenge in swine waste management in large farms is thus effluent processing. As a consequence, extensive research has been carried out on the use of novel technologies to adequately handle effluent generation at a farm level. There are three primary approaches for swine waste remediation at farm scale: solid/liquid separation methods, solid fraction treatments, and liquid fraction treatments. A list of available technologies can be found in Table \ref{table2}. From those mentioned, the most common process to improve the characteristics of SW is anaerobic digestion (AD). AD is the biotechnological application of methanogenesis, a process which uses microorganisms to achieve the complete degradation of organic matter in the absence of oxygen. In AD, complex microbial communities mediate the conversion process in a series of consecutive steps (hydrolysis, acidogenesis, acetogenesis and methanogenesis) and yield a mixture of gaseous products comprising mainly CH\textsubscript{4} and CO\textsubscript{2} called biogas. The presence of CH\textsubscript{4} in high quantities (tipically 60-65 \%) in biogas has established its use as a fuel within many applications \cite{Mata_Alvarez_2014, O_Flaherty_2010}. Despite contributing positively to the reduction of gas emissions and water and soil pollution, AD does not change the overall N/P ratio of the effluent, and it only impacts N availability. With this in mind, electrochemical methods like electrooxidation (EO) have recently been explored for the complete mineralization of organics (biodegradable and refractory), as well as nitrogen compounds, usually as a post-treatment step on digested effluents. EO achieves the removal of these pollutants using both the direct and indirect effects of passing an electric current through the wastewater. Briefly, EO can follow two mechanisms: direct electron transfer in the potential region before oxygen evolution (direct oxidation); and indirect electron transfer via electrogenerated reactive oxygen species or other electrogenerated oxidants, in the potential region of oxygen evolution. Electrochemical approaches to swine wastewater remediation have the technical benefits of safety (operation at ambient temperature and pressure), versatility (modular design and easy adaptation to different influent compositions and volumes), environmental compatibility (in general, no need for chemicals addition and no waste generation) and efficiency; however, cost-effectiveness due to relatively high energy and electrode costs, as well as concerns with the secondary generation of toxic chlorinated compounds, can be considered drawbacks depending on process design and engineering \cite{Rajeshwar_1994, Mart_nez_Huitle_2006, Radjenovic_2015}.
