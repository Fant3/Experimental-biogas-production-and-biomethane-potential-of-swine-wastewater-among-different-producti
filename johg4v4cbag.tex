\section{Conclusion}
In this study liquid swine wastes from a farrow to finish farm were anaerobically digested under mesophilic conditions. In general, the characteristics of gestation, weaners, fattening, and farrowing were significantly different due to different feed strategy and nutrient digest- ibility at different growth stages. Fattening wastewater presented higher solids and organics contents, while farrowing effluent streams were considered low strenght due to mild physicochemical characteristics. The specific methane yields for each substrate wastewater ranged from 293.9 mLCH\textsubscript{4}.gVS added\textsuperscript{-1} for gestation and 172.4 mLCH\textsubscript{4}.gVS added\textsuperscript{-1} for weaners. Fattening was These values, per se, are encouraging for on farm energy recovery and may provide an important contributor to alleviate the increasing energy demand within the industry. An HRT of 41–53 days is recom-mended for effective treatment of FW silages under continuous operations. The modified Gompertz model fitted the experimental results better than a first-order kinetic model