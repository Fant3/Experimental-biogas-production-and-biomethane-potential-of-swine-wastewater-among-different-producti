\section{Conclusion}
In this study, swine wastewaters from a farrow to finish farm were anaerobically digested under mesophilic conditions. In general, the characteristics of gestation, weaners, fattening, and farrowing wastewaters were significantly different due to different feed strategy and nutrient digestibility at different growth stages. Fattening wastewater presented higher solids and organics contents, while farrowing effluent streams were considered low strength due to mild physicochemical characteristics. The specific methane yields for each substrate wastewater was successfully determined gravimetrically and ranged from 293.9 mLCH4.gVS added\textsuperscript{-1} for gestation and 172.4 mLCH4.gVSadded\textsuperscript{-1} for weaners. Farrowing wastewater presented no detectable biogas productionin the studied conditions probably due to very low organics content. These values are encouraging for on-farm energy recovery and may provide an important contributor to
alleviate the increasing energy demand within the industry. The Logistic and Gompertz model fitted the experimental results well and may provide valuable knowledge for the treatment of specific swine wastewater streams at different growth stages within a perspective of producing biogas. With this in mind, experimental data suggest an HRT between 28–36 days for an effective treatment under continuous operations. However, as indicated by the partial biodegradability of the waste streams, AD should not beviewed as a complete solution and should be integrated with other treatment methods,
either as pre-treatment or post-treatment.