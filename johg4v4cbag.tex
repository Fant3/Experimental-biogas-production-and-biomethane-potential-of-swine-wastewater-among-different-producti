\section{Conclusion}
In this study liquid swine wastes from a farrow to finish farm were anaerobically digested under mesophilic conditions. In general, the characteristics of gestation, weaners, fattening, and farrowing were significantly different due to different feed strategy and nutrient digest- ibility at different growth stages. Fattening wastewater presented higher solids and organics contents, while farrowing effluent streams were considered low strenght due to mild physicochemical characteristics. The specific methane yields for each substrate wastewater was sucsecsfully determined gravimetrically  and ranged from 293.9 mLCH\textsubscript{4}.gVS added\textsuperscript{-1} for gestation and 172.4 mLCH\textsubscript{4}.gVS added\textsuperscript{-1} for weaners. Farrowing wastewater presented no detectable biogas production in the studied conditions probably due to their very low organics content as measured by VS and COD; in reality, however, some biogas was likely produced during the experiment but in such a low amount that it was not measureable gravimetrically. These values, per se, are encouraging for on farm energy recovery and may provide an important contributor to alleviate the increasing energy demand within the industry. The Logistic and Gompertz model fitted the experimental results well and may provide valuable knowledge for the treatment of specific swine wastewater streams at different growth stages within a perspective of producing biogas. With this in mind, the experimental data obtained suggests an HRT between 28–36 days for an effective treatment under continuous operations. 