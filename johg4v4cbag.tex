\section{Conclusion}
In this study liquid swine wastes from a farrow to finish farm were anaerobically digested under mesophilic conditions. The characteristics of ge were signifi-
cantly different due to different feed strategy and nutrient digest- ibility at different growth stages The specific methane yields for each substrate wastewater, ranging from 248.0 mLCH\textsubscript{4}.gVS added\textsuperscript{-1}, 293.9 mLCH\textsubscript{4}.gVS added\textsuperscript{-1}, and 172.4 mLCH\textsubscript{4}.gVS added\textsuperscript{-1} for fattening, gestation, and weaners, are encouraging for on farm energy generation and may provide an important contributor to alleviate the increasing energy demand within the industry. A minimum HRT of 21–24 days is recommended to prevent biomass washout and an HRT of 41–53 days is recom- mended for effective treatment of FW silages under continuous operations. The modified Gompertz model fitted the experimental results better than a first-order kinetic model