\subsection{SW composition and physicochemical properties}
An updated knowledge of SW composition is a fundamental resource in order to determine if the effluent should be regarded as a raw material for  valuable products (e.g. organic fertilizer, energy carrier) or as a risk to the surrounding environment. Generally, SW contains mainly suspended solids, organic matter, and micronutrients arising from animal food digestion, i.e., animal excreta in the form of urine and faeces. In addition to solid and liquid excrements, SW derived from pig houses contain the remains of food and bedding (straw, sand, sawdust, etc.), as well as washing waters. As such, SW wet chemistry can be very dynamic as specific concentrations of these components may vary depending on animal feed, as well as other factors such as the housing system, manure management practices and environmental regulations. Wide variations intra-farms and inter regions are, for example, very common and a broad range of values have been reported in the literature. Nevertheless, a complete picture of SW composition can be given with some detail. Tipically, factors of importance in SW characterization include pH, and concentration of organic and inorganic components. Moreover, some physical and electrochemical properties of the particles are also important for optimizing the use of different processing technologies. In general, SW  is characterized by a low total solids content ($<$12\%), with about 70-75\% comprising organic materials \cite{Christensen_2009}. Carbohydrates comprise the largest fraction of the organics presents, followed by proteins, lipids, lignin and VFAs (Volatile Fatty Acids) \cite{Jensen_2013}. Indirect measures of organic compounds presence include BOD\textsubscript{5} and COD and thus SW is generally characterized by high values of both parameters (6500-7200 mg.\textsuperscript{-1}; 4684-63724 mg.\textsuperscript{-1}, respectively) depending on farm circumstances \cite{Hai_2015, C_rdoba_2016, Villamar_2011}. Regarding the pH, SW is usually neutral (6.8-7.3) due to the presence of short-chain VFAs . Particle charge and ionic strength, which affect the electrical potential around the particles, is also generally high due to the presence of salts, i.e, conductivity $>$ 10 mS.cm\textsuperscript{-1}, with organic particles often having a negative surface charge (-mV) \cite{Hjorth_2010}. SW often contains high amounts of total N (1062-2222 mg.\textsuperscript{-1}) and total P (32-181 mg.\textsuperscript{-1}) as pig usually excrete a relatively high proportion of N and P intake \cite{Hai_2015, C_rdoba_2016, Villamar_2011}. Nitrogen occurs in organic combinations (proteins, amino-acids, urea, uric acid) and mineral combinations (ammonia, nitrates) and is usually present in the form of dissolved ammonia (948-1558 mg.\textsuperscript{-1}) \cite{Hai_2015, C_rdoba_2016, Villamar_2011}. P compounds are generally present in the particulate fraction of SW and less than 30\% is dissolved in the liquid phase. Of the dissolved P, more than 80\% is in the form of orthophosphate (PO\textsubscript{4}\textsuperscript{3−} (aq)) \cite{Christensen_2009}. SW also contains high amounts of Cu and Zn and other trace metals such as Fe, Mn, Cd and B.
