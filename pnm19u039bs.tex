\subsection{Kinetic modelling}
Biogas production data was fitted against three classical model formulas for microbial growth (Logistic, Gompertz, and Richards) using grofit R package \cite{Kahm_2010}. Grofit software was developed to describe many biological growth curves obtained under different conditions but has rarely been applied to biogas production modelling. It enables a semiautomatic fitting of the parameters describing most growth processes which typically present a phase of slow gas production (lag phase) followed by a period of rapid gas production and then by a stationary phase. In this study, the fitting of modified sigmoidal models (see \cite{Zwietering1990} for details) to the cumulative biogas production curves with respect to time allows the determination of parameters with biological meaning such as the maximum biogas yield ($A$), the maximum rate of biogas production ($μ_m$), as well as the length of the lag phase ($λ$), as described in equations (8), (9), and (10) for the different models \cite{Zwietering1990,Ware_2017,Alta__2009}. The best fitting model is automatically chosen by grofit using the Akaike criterion \cite{Hasenbrink_2006}.

Logistic model

\(y=\frac{A}{\left(1+\exp\left[\frac{4\mu_m}{A\left(\lambda-t\right)+2}\right]\right)}\)

Gompertz model

\(y=A.\exp\left[-\exp\left(\frac{\mu_me}{A}\left(\lambda-t\right)+1\right)\right]\)

Richards model

\(y=\frac{A}{\left(1+d\ .\exp\left(1+d\right).\exp\left[\frac{\mu_m}{A}.\ \left(1+d\right)\left(1+\frac{1}{d}\right).\left(\lambda-t\right)\right]\right)^{\left(-\frac{1}{d}\right)}}\)

where $y$ is the cumulative biogas production (mL.gVS\textsuperscript{−1}), $A$ is the maximum biogas production potential (mL.gVS\textsuperscript{−1}), $μ_m$ the maximum rate of methane production rate (mL.gVS\textsuperscript{−1})d\textsuperscript{−1}), $e$ is Euler's constant, $λ$: lag-phase (days), $t$ is the incubation time (days), and $d$ the shape coefficient.