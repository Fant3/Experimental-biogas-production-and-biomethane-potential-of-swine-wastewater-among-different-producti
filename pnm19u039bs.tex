\subsection{Kinetic modelling}
Biogas production data was fitted against four classical model formulas for microbial growth (Logistic, Gompertz, modified Gompertz and Richards) using grofit R package \cite{Kahm_2010}. Grofit software was developed to describe many biological growth curves obtained under different conditions but has rarely been applied to biogas production modelling. It enables a semiautomatic fitting of the parameters describing most growth processes which typically present a phase of slow gas production (lag phase) followed by a period of rapid gas production and then by a stationary phase. In this study, the fitting of modified sigmoidal models (see \cite{Zwietering1990} for details) to the cumulative biogas production curves with respect to time allows the determination of parameters with biological meaning such as the maximum biogas yield (A), the maximum rate of biogas production (μ), as well as the length of the lag phase (λ), as described in equations (8), (9), and (10) for the dif. The best model is automatically chosen by grofit using the Akaike criterion \cite{Hasenbrink_2006}.