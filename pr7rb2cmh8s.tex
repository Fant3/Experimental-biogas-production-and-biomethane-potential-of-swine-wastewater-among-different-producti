\subsection{Kinetic model}
Cumulative biogas yield obtained from the experimental assays was used to fit a kinetic model of biogas production with the modeled results found to have the best fit plotted against the experimental cumulative biogas production in Fig.4, Fig.5, and Fig.6 depending on the growth stage. The parameters determined in the optimization process as well as statistical indicators and the difference between the predicted and experimental biogas production are presented in Table 6.
Concerning the kinetics of the process, the biogas production of the microorganisms typically displayed a lag phase in which the bacterial cells modified their physiological state to adapt to the environment inside the reactor. The developed biosystem then multiplied and degraded volatile solids into biogas with an exponential phase and a final stationary phase in a reverse L-shape curve typical of simple organic substrates. As seen in the figures, the curves for gestation, fattening and weaners wastewaters all show a similar trend upon visual inspection, i.e., the chosen model adequately describes the biogas production from the tested effluents. 
Among the tested models, the best fit was obtained from the Logistic and Gompertz equation for gestation, and weaners and fattening wastewaters, respectively. The λ, μ\textsubscript{m} and A  parameters of different models presented some differences for each effluent (Table 6). Calculated lag time (λ) was found to be about 2 days for fattening wastewaters, varying from 6 and almost 10 days for gestation and weaners waste streams. No significant difference in λ values was found for gestating and fattening wastewaters, but the lag phase for weaners was significantly higher than that for fattening. The high bioavailability of readily degradable organics in the latter wastewaters can be the reason for the shorter lag phase. The highest maximum biogas production (μ\textsubscript{m} rate) values were estimated for gestting sows followed by the CM and the lowest was for the GM. No significant difference in K and lm values were found for the CM and SM but the predicted Rm value for the SM was significantly higher than that for the CM.