\subsection{Kinetic model}
Cumulative biogas yield obtained from the experimental assays was used to fit a kinetic model of biogas production with the modeled results found to have the best fit plotted against the experimental cumulative biogas production in Fig.4, Fig.5, and Fig.6 depending on the growth stage. The parameters determined in the optimization process as well as statistical indicators and the difference between the predicted and experimental biogas production are presented in Table 6.
Concerning the kinetics of the process, the biogas production of the microorganisms typically displayed a lag phase in which the bacterial cells modified their physiological state to adapt to the environment inside the reactor. The developed biosystem then multiplied and degraded volatile solids into biogas with an exponential phase and a final stationary phase in a reverse L-shape curve typical of simple organic substrates. As seen in the figures, the curves for gestation, fattening and weaners wastewaters all show a similar trend upon visual inspection, i.e., the chosen model adequately describes the biogas production from the tested effluents. 

Among the tested models, the best fit was obtained from the Logistic and Gompertz equation for gestation, and weaners and fattening wastewaters, respectively. One-way ANOVA has shown some differences between λ, μ\textsubscript{m} and A  parameters for each production stage (Table 6). Calculated lag time (λ), for example, was found to be about 2 days for fattening wastewaters, varying from 6 and almost 10 days for gestation and weaners waste streams. No significant difference in λ values was found for gestating and fattening wastewaters, but the lag phase for weaners was significantly higher than that for fattening. The high bioavailability of readily degradable organics in the latter wastewaters can be the reason for the shorter lag phase. The highest and lowest maximum biogas production (μ\textsubscript{m} rate) values were estimated for gestating and fattening sows (26.46-12.87 mL.gVS added\textsuperscript{-1}.d\textsuperscript{-1}) with the values for gestation wastewater being statistically different than that of weaners and fattening as revealed by Tukey’s HSD test. Finally, maximum biogas yield (A) has presented statistically significant differences between production stages with higher values for wastewaters from gestation (458.73 mL.gVS added\textsuperscript{-1}) followed by fattening and weaners (347.33 mL.gVS added\textsuperscript{-1} and 239.94 mL.gVS added\textsuperscript{-1} respectively). The statistical indicators (R\textsuperscript{2} values, F-statistic and p-value) are also shown in respective figures. R\textsuperscript{2} ranged between (0.9970–0.9957) showing that the traditional models tested are very useful in providing an accurate description of the experimental data. The kinetic parameters determined from the modelling provided further insights into the results of the experimental assays, in particularly the biodegradation patterns of the different wastewaters. During the first 5-10 days kinetics is better for fattening wastewaters with the low λ value suggesting the presence of readily biodegradable organic matter. However, after 10 days the behaviour changes and gestation effluents biogas productivity increases as revealed by the maximum biogas production rate two times higher when compared to. This performance indicates the presence of slowly or less readily  biodegradable organics which makes the process slower at the beginning. Alternatively, the microorganisms may also not be acclimated to the wastewater. For weaners efluents, the extended lag phase and the low biogas producti