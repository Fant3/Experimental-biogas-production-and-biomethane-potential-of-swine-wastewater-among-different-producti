\subsection{Kinetic model}
Cumulative biogas yield obtained from the experimental assays was used to fit a kinetic model of biogas production with the modeled results found to have the best fit plotted against the experimental cumulative biogas production in Fig.4, Fig.5, and Fig.6 depending on the growth stage. The parameters determined in the optimization process as well as statistical indicators and the difference between the predicted and experimental biogas production are presented in Table 6.
Concerning the kinetics of the process, the biogas production of the microorganisms typically displayed a lag phase in which the bacterial cells modified their physiological state to adapt to the environment inside the reactor. The developed biosystem then multiplied and degraded volatile solids into biogas with an exponential phase and a final stationary phase in a reverse L-shape curve typical of simple organic substrates. As shown in the figures, the curves for Gestation, Fattening and Weaners wastewater all show a similar trend upon visual inspection    