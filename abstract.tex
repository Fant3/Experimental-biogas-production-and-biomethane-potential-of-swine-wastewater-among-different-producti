Effluent streams originating from swine production can be a major cause of point-source pollution and very dangerous for the surrounding environment. In this work, biogas and biomethane potential of swine-derived effluents was assessed under mesophilic conditions among different production stages (Farrowing, Gestation, Weaners, and Fattening) in order to investigate the suitability of anaerobic digestion as a treatment technology and energy recovery tool. Anaerobic biodegradability and kinetic modelling of the different wastewaters were also evaluated in order to better understand the degradation patterns of each waste stream. The specific methane yields for each substrate wastewater were successfully determined gravimetrically and ranged from 293.9 ± 31.6 mLCH4.gVS added-1, and 172.4 ± 1.62 mLCH4.gVS added-1 for Gestation, 248.0 ± 246.6 mLCH4.gVS added-1 for Fattening, and 172.4 mLCH4.gVS added-1 for Weaners. Regarding anaerobic biodegradability 