\subsection{Experimental BMP}
The cumulative biogas production, daily biogas production, and methane content in the biogas produced during the digestion of SW from different production stages under mesophilic conditions are presented in Fig.1, Fig.2, and Fig.3. Gestation wastewater presented the largest biogas potential after 54 days (500.35 (0.13) ml.g VS added\textsuperscript{-1}) followed by fattening (385.50 (333.71) ml.g VS added\textsuperscript{-1}) and weaners (250.42 (0.37) ml.g VS added\textsuperscript{-1}) wastewater. Farrowing wastewater showed no detectable biogas production in the studied conditions due to their very low organic content as measured by VS; however, some biogas was likely produced during the experiment but in such a low amount that it was not measureable gravimetrically.  ANOVA on the cumulative biogas production showed no statistically significant differences between different farming.\\
With respect to daily biogas production rates, peak values were calculated to be 26.0, 14.4, and 18.3 ml.gVS added\textsuperscript{-1} after 14, 14, and 9 days for swine wastewater from gestation, weaners, and fattening stages, respectively.