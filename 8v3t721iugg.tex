\subsection{Experimental biogas and biomethane production}
The cumulative biogas production, daily biogas production, and methane content in the biogas produced during the digestion of SW from different production stages under mesophilic conditions are presented in Fig.1, Fig.2, and Fig.3.\\

Biogas production started immediately on the first 5 days of digestion in all reactors but those containing weaners wastewaters. Biogas production was very low in these digesters until day 9 when the gas production began to sustainably increase until day 14. Thereafter, biogas production remained relatively constant until day 21 before slowly decreasing and ceasing production in day 54. In the reactors containing gestation and fattening wastewaters, biogas production rapidly increased until days 14 and 9, respectively, before continuously decrease until the end of the experiment. Peak values for the daily biogas production rate were calculated to be 25.0 ± 0.78, 13.8 ± 0.74, and 17.9 ± 1.49 ml.gVS added\textsuperscript{-1} after 14, 14, and 9 days for swine wastewater from gestation, weaners, and fattening stages, respectively.
Farrowing wastewater showed no detectable biogas production in the studied conditions due to their very low organics content as measured by VS and COD; in reality, however, some biogas was likely produced during the experiment but in such a low amount that it was not measureable gravimetrically.\\
Gestation wastewater presented the largest average biogas potential after 54 days (461.3 ± 33.7) ml.gVS added\textsuperscript{-1}) followed by weaners (245.8 ± 3.16) ml.g VS added\textsuperscript{-1}) wastewater. Fattening wastewater, in turn, presented a biogas potential of 348.6 ± 342.54 ml.gVS added\textsuperscript{-1}) after 61 days.  ANOVA on the cumulative biogas production showed no statistically significant differences between different stages. As shown in Fig. 1, most of the final biogas yields were obtained in the first 28-36 days of digestion for each production stage, specifically 89.8\%, 92.7\%, and 84.4\% for fattening, gestation and weaners wastewater, after 28, 36, and 36 days, respectively. These values may be used as suitable hydraulic retention times in the treatment of each wastewater in a continuous system \cite{Li_2013}.

The methane content in the biogas produced presented a similar trend in all digesters. Fattening wastewater presented the higher methane contents with values increasing from 28.85\% ± 7.00 to 60.2\% (8.34) between day 5 and day 21. With respect to gestation wastewater, the methane content increased from 8.70\% (3.54) in day 5 to a maximum value of 47.1\% after about 36 days. In the digesters containing weaners wastewater, a peak value of 45.45\% (0.07) methane was achieved in day 36, increasing from 4.6\% (0.00) in day 5 of digestion. The specific methane yields for each substrate wastewater were 258.58 (258.09) mlCH\textsubscript{4}.gVS added\textsuperscript{-1}, 299.66 (35.85) mlCH\textsubscript{4}.gVS added\textsuperscript{-1}, and 174.31 (2.40) mlCH\textsubscript{4}.gVS added\textsuperscript{-1} for fattening, gestation, and weaners wastewater. From these results, it is apparent that the digesters containing fattening wastewater showed a very high variability between duplicates. Despite similar sampling procedures during collection, issues may have occured during sample management prior to storage resulting in uneven samples for digestion. In fact,  according to previous research, representative and uniform samples cannot be obtained without using proper agitation methods during sample collection and storage \cite{Zhu_2004}. Nevertheless, when consistently comparing these results with typical methane values obtained by other researchers (see Table 5), similarities can be found suggesting that the overall yield obtained for fattening wastewater is  reasonable \cite{Guo_2012,Zhang_2014}.