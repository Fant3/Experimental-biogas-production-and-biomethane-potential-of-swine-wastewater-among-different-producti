\subsection{Theoretical BMP and biodegradability}
The amount of biogas produced from the degradation of a specific sample, in the case SW from different production stages, can be theoretically estimated using its elemental composition and COD characterization. From the work of Symons and Buswell from the 1930s \cite{Symons_1933}, a basic understanding of the biochemical oxidation-reduction reactions occurring during AD was reached, allowing for the calculation of the expected methane production of a known biochemical compound. According to the method, which is known as Buswell equation, the elemental composition of the substrate (C, H, O, N, S) is used to derive an empirical formula of the substrate and calculate the volumes of CH\textsubscript{4} and CO\textsubscript{2} resulting from its degradation. The formula, however, presumes that only biodegradable matter is present and that all electrons donated are exclusively used for metabolic energy, neglecting the energy demand of microbial populations (cellular synthesis)\cite{Labatut_2011,Lesteur_2010}. Total stoichiometric conversion of the organic compounds is therefore assumed, resulting in overestimation of methane potential. Nonetheless, the method can provide fast and easy indication as to the biogas composition of a given substrate \cite{Ware_2016}.\\

\(C_cH_hO_oN_nS_s+\frac{(4c-h-2o+3n+2s)}{4}H_2O→\frac{(4c-h+2o+3n+2s)}{8}CO_2+\frac{(4c+h-2o-3n-2s)}{8}CH_4+nNH_3+sH_2S\)\\

\(BMP_{th}=\frac{22.4\left(\frac{c}{2}+\frac{h}{8}-\frac{o}{4}\right)}{12c+h+16o}\ \left(STP\ \frac{L\ CH_4}{g\ VS}\right)\)\\

COD concentration can also be used to estimate the theoretical CH\textsubscript{4} yield as COD indirectly measures the amount of organic matter present in a given substrate. This method is based on the assumption that 1 mole of methane requires 2 moles of oxygen to oxidise carbon to carbon-dioxide and water \cite{Jingura_2017,Nielfa_2015}. 
\\
\(BMP_{thCOD}=\frac{n_{CH_4}RT}{pVS_{added}}\)

\(n_{CH_4}=\frac{COD}{64\ \left(g.mol^{-1}\right)}\)
\\
where $BMP_{thCOD}$ is the theoretical production in ideal conditions, $R$ is the gas constant (R= 0.082 atmL.mol\textsuperscript{-1}K\textsuperscript{-1}), $T$ is the temperature of the Reactor (308 K), p is the atmospheric pressure (1 atm), $VS_{added}$ (g) are the volatile solids of the substrate and $nCH_4$ is the amount of molecular methane (mol).

Biodegradability of a given substrate can 

\subsection{Data analysis}

Statistical analysis of the experimental results with respect to samples characterization and BMP potential was carried out by means of R software \cite{nokey_e6883}. A one-way ANOVA test was implemented to evaluate the different growth stages, after which post hoc multiple comparison was carried out by means of the Tukey HSD test at the 95\% confidence level. All values correspond to the mean of two independent replicates (n =2) ± standard deviation (SD).

