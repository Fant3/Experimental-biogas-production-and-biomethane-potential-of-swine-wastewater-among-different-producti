\section{Materials and Methods}
\subsection{Farm description, substrates and inoculum}
The swine wastewater used in the study was obtained from a large scale pig farm located in Alentejo, Portugal. The farm operates in a closed cycle, i.e, pigs are raised in separated breeding units which are classified as gestating sows (GS), swine nurseries/farrowing (SN), weaners (W) and growing fattening (GF). The management of wastewater on farm follows a traditional pathway based on sequential process comprising a solid-liquid separation unit and biological treatment (aerobic lagoons). Swine wastewater samples were collected from drainpipes that carried wastewater from the slatted floors inside buildings to external tanks before any treatment was applied. Each sample was classified according to the different growth stages and stored at - 20\textdegree C until usage in the present work. According to previous research,representative samples cannot be obtained without using proper agitation methods. \cite{Zhu_2004}, for example, compared different manure sampling procedures (vertical mixing, horizontal or no mixing) and found out that randomly taking a sample without agitation cannot guarantee homogenization and thus should not be used. On the other hand, both agitation sampling procedures worked well in homogenizing samples with no statistically significant difference between them.  Therefore, sampling procedure used in this work included agitationin order to obtain a uniform sample for compositional analysis.
\subsection{Analytical methods}
Substrates were characterized based on selected physicochemical and biochemical parameters relevant for the present work. Total solids (TS) and volatile solids (VS) were determined according to Standard Methods, sections 2540B and 2540E, respectively (APHA, 1995). Chemical oxygen demand (COD) was based on the colorimetric dichromate closed reflux method from section 5220D of Standard Methods (APHA, 1995), using HACH digestion vials (HACH Co., Loveland, CO). 10‐day biochemical oxygen demand (BOD) was determined using the HACH BODTrakTM (HACH Co., Loveland, CO). Total Kjeldahl nitrogen (TKN) determination was conducted according to the HACH Nessler method 8075 (HACH, 2003). Ammonia‐N was determined based on the HACH Salicylate method 10031 (HACH, 2003). 
\subsection{BMP assay experimental procedure}
The experimental procedure followed in this study was based on the principles described by \cite{Owen_1979} and later revised by other researchers \cite{Angelidaki_2009,Holliger_2016}. Specifically, BMP assays were carried out in serum bottles (Schott Duran, Germany) of 1000 mL, with a working liquid volume of 600 mL and a headspace volume of 400 mL. Bottles contained appropriate quantities of substrate/inoculum in a ratio of 3.0 g VS substrate/g VS inoculum, based on the specific characteristics of the effluent. The bottles were flushed with nitrogen gas, sealed with 5mm thick silicone discs (Schott Duran, Germany) and closed by a plastic screw cap (Schott Duran, Germany). Each bottle was punctured by a hypodermic needle connected to a capillary tube equipped with a 3-way valve and placed in an incubator at a constant mesophilic temperature (37 °C). Biogas production was measured indirectly by mass loss (gravimetric method) as detailed by \cite{Hafner_2015}. The mass of each reactor was determined to the nearest 10 mg before and after any period of biogas production, during which time biogas was removed/collected in gas bags by puncturing with a hypodermic needle until atmospheric pressure/equilibrium was reached. The collected biogas was immediately characterized by a gas analyzer which allows to measure the volume percentages of CO\textsubscript{2},CH\textsubscript{4},O\textsubscript{2},H\textsubscript{2}S, CO in the mixture. Mixing was performed manually during the incubation period at regular times. BMP assay was ended when biogas production ceased or reached a plateau. All experimental assays were performed in duplicate and control/blank bottles with only inoculum/water were included in order to correct the obtained methane production.  All values are expressed at standard temperature and pressure.
\subsection{Theoretical BMP}
The amount of biogas produced from the degradation of a specific sample, in the case SW  can be theoretically estimated using the Buswell equation.


\subsection{Experimental design and data analysis}


