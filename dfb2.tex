\section{Materials and Methods}
\subsection{Farm description, substrates and inoculum}
The swine wastewater used in the study was obtained from a large scale pig farm located in Alentejo, Portugal. The farm operates in a closed cycle, i.e, pigs are raised in separated breeding units which are classified as gestating sows (GS), swine nurseries/farrowing (SN), weaners (W) and growing fattening (GF). The management of wastewater on farm follows a traditional pathway based on sequential process comprising a solid-liquid separation unit and biological treatment (aerobic lagoons). Swine wastewater samples were collected from drainpipes that carried wastewater from the slatted floors inside buildings to external tanks before any treatment was applied. Each sample was classified according to the different growth stages and stored at - 20\textdegree C until usage in the present work. According to previous research,representative samples cannot be obtained without using proper agitation methods. \cite{Zhu_2004}, for example, compared different manure sampling procedures (vertical mixing, horizontal or no mixing) and found out that randomly taking a sample without agitation cannot guarantee homogenization and thus should not be used. On the other hand, both agitation sampling procedures worked well in homogenizing samples with no statistically significant difference between them.  Therefore, sampling procedure used in this work included agitationin order to obtain a uniform sample for compositional analysis.
\subsection{Analytical methods}
Substrates were characterized based on selected physicochemical and biochemical parameters relevant for the present work. Total solids (TS) and volatile solids (VS) were determined according to Standard Methods, sections 2540B and 2540E, respectively (APHA, 1995). Chemical oxygen demand (COD) was based on the colorimetric dichromate closed reflux method from section 5220D of Standard Methods (APHA, 1995), using HACH digestion vials (HACH Co., Loveland, CO). 10‐day biochemical oxygen demand (BOD) was determined using the HACH BODTrakTM (HACH Co., Loveland, CO). Total Kjeldahl nitrogen (TKN) determination was conducted according to the HACH Nessler method 8075 (HACH, 2003). Ammonia‐N was determined based on the HACH Salicylate method 10031 (HACH, 2003). 
\subsection{BMP assay experimental procedure}
The experimental procedure followed in this study was based on the principles described by \cite{Owen_1979} and later revised by other researchers \cite{Angelidaki_2009,Holliger_2016}. Specifically, BMP assays were carried out in serum bottles (Schott Duran, Germany) of 1000 mL, with a working liquid volume of 600 mL and a headspace volume of 400 mL. Bottles contained a total organic matter concentration of x g VS/L and the substrate/inoculum (S/I) ratio was 3.0 g VS substrate/g VS inoculum, based on the characteristics of swine wastewater. The bottles were flushed with nitrogen gas, sealed with 5mm thick silicone discs (Schott Duran, Germany) and held tightly to the bottle head by a plastic screw cap (Schott Duran, Germany). Each bottle was punctured by a hypodermic needle connected to a capillary tube equipped with a 3-way valve and placed in an incubator at a constant mesophilic temperature (35 °C). Biogas production was measured indirectly by the pressure increase in the headspace volume using a calibrated pressure transducer sensor (error ± 1.5\%) connected to one of the valve ends. After each measurement, biogas was collected from the reactor's headspace using a polycarbonate syringe of 20 mL connected to the 3-way valve and immediately analyzed by gas chromatography; if any, the remaining biogas was released until atmospheric pressure was reached in the reactors. Mixing was performed manually during the incubation period at regular times, usually every other day. BMP assay was ended when biogas production ceased or reached a plateau. All experimental assays were performed in duplicate and one contro bottles without the substrate were included and subtracted from cumulative methane production.  All values are expressed at standard temperature and pressure.
\subsection{Theoretical BMP}

\subsection{Experimental design and data analysis}


