\section{Materials and Methods}
\subsection{Farm description, substrates and inoculum}
The swine wastewater used in the study was obtained from a large scale pig farm located in Alentejo, Portugal. The farm operates in a closed cycle, i.e, pigs are raised in separated breeding units which are classified as gestation (gestation sows until right before giving birth), swine nurseries/farrowing (gestation sows a few days before giving birth to until weaning), weaners (from farrowing to 28 kg of bw) and growing fattening (from 28 kg to 110 kg of bw). The management of wastewater on farm follows a traditional pathway based on sequential process comprising a solid-liquid separation unit and biological treatment (aerobic lagoons). 
Swine wastewater samples were collected from drainpipes that carried wastewater from the slatted floors inside buildings to external tanks before any treatment was applied. Each sample was classified according to the different growth stages and stored at - 20\textdegree C until usage in the present work. The inoculum was fresh cow manure obtained from a local farm in Portalegre, Portugal, and had the following characteristics: pH 7.3 , 20.44  0.45 \%TS and 60.96 1.39\% VS (\%TS).

\subsection{Analytical techniques}
Substrates were characterized based on selected physicochemical and biochemical parameters relevant for the digestion process. pH, EC, ORP and TDS were measured prior to sample storage using a Hanna pH meter. TS and TVS were determined gravimetrically using standard methods by placing the sample in the oven for at least 24h at 105\textdegree C and the obtained dried mass subsequently placed in a furnace at 550\textdegree C for 2h to obtain the volatile solids content as a fraction of the total solid (\%TS). Chemical oxygen demand (COD) was determined by the standard dichromate method using digestion vials. After chemical digestion at 150\textdegree C for 2h (Aqualytic AL32 thermoreactor), samples were returned to ambient temperature and COD determination achieved using a photometer (Aqualytic COD vario – PC compact) \cite{InternationalOrganizationforStandardization2002}. 5‐day biochemical oxygen demand (BOD\textsubscript{5}) was determined using a Aqualytic A311 sensor system based on the manometric principle and following the standard methods 5210D procedure. Ultimate analysis to determine carbon (C), nitrogen (N), hydrogen (H) and Sulfur (S) contents was performed for the generation of stoichiometric description of the wastewater. Samples were firstly oven dried at 105\textdegree C until constant weight was obtained and then analyzed using a Flash Thermo CHNS-O 2000. Oxygen content (O) was calculated by difference, O\% = 1-C\% - N\% - H\% - S\% - ash\%).

\subsection{BMP assay experimental procedure}
The experimental procedure followed in this study was based on the principles described by \cite{Owen_1979} and later revised by other researchers \cite{Angelidaki_2009,Holliger_2016}. Specifically, BMP assays were carried out in serum bottles (Schott Duran, Germany) of 1000 mL, with a working liquid volume of 600 mL and a headspace volume of 400 mL. Bottles contained appropriate quantities of inoculum/substrate in a ratio of 3.0 g VS inoculum/g VS substrate. The bottles were flushed with nitrogen gas, sealed with 5mm thick silicone discs (Schott Duran, Germany) and closed by a plastic screw cap (Schott Duran, Germany). Each bottle was then placed in an incubator at a constant mesophilic temperature (37 °C). Biogas production was measured indirectly by mass loss (gravimetric method) as detailed by \cite{Hafner_2015}. The mass of each reactor was determined to the nearest 10 mg before and after any period of biogas production, during which time biogas was removed/collected in gas bags by puncturing with a hypodermic needle until atmospheric pressure/equilibrium was reached. The collected biogas was immediately characterized by a hand-hel gas analyzer (GasData GFM406) which allows to measure the volume percentages of CO\textsubscript{2},CH\textsubscript{4},O\textsubscript{2},H\textsubscript{2}S, CO in the mixture. Mixing was performed manually during the incubation period at regular times. BMP assay was ended when biogas production ceased or reached a plateau. All experimental assays were performed in duplicate and control/blank bottles with only inoculum/water were included in order to correct the obtained methane production.  All values are expressed at standard temperature and pressure.
