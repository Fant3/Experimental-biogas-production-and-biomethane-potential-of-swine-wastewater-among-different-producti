\subsection{Theoretical BMP and methane-based biodegradability}
In order to evaluate the biodegradability of the tested wastewaters, the specific methane yield obtained in the BMP assay was compared with the theoretical methane yield estimated using its elemental composition and organics content measured via COD. Table 6 presents theoretical results (BMP\textsubscript{th}, BMP\textsubscript{thCOD}, and CH\textsubscript{4th} ) using both methodologies, as well as experimental results (specific methane yields and CH\textsubscript{4exp}) and  biodegradability.  Theoretical methane potential calculation results showed that wastewaters from the fattening stage (BMP\textsubscript{th} = 653.46, BMP\textsubscript{thCOD} = 991.47) had a higher thoeretical potential than gestation (BMP\textsubscript{th} = 487.65, BMP\textsubscript{thCOD} = 471.69) or weaners (BMP\textsubscript{th} = 374.98, BMP\textsubscript{thCOD} = 618.84) wastewater. In general, the theoretical results obtained from elemental composition presented a better agreement with the experimental results. According to this methodology, the biodegradability of gestation, fattening, and weaners wastewater was 61.45\%, 39.57\%, and 46.49\%. Despite not having the higher theoretical potential, gestation wastewater were lower when compared to gestation wastewater due to lower biodegradability. Specifically, with The results are in the range to those obtained by other authors \cite{Zhang_2014} who experimented with similar swine wastewater streams.