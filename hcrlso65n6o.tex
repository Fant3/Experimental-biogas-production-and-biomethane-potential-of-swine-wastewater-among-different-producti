\subsection{Theoretical BMP and biodegradability}
In order to evaluate the biodegradability of the tested wastewaters, the specific methane yield obtained in the BMP assay was compared with the theoretical methane yield estimated using its elemental composition and organics content measured via COD. Table 6 presents theoretical results (BMP\textsubscript{th}, BMP\textsubscript{thCOD}, and CH\textsubscript{4th} ) using both methodologies, as well as experimental results (specific methane yields and CH\textsubscript{4exp}) and  biodegradability.  Theoretical methane potential (TMP) calculation results
showed that K (TMP = 725 mL g−1 VSadded) had a higher TMP than CS (TMP = 470 mL g−1 VSadded) or CM (TMP = 617 mL g−1 VSadded). Therefore, KW had the highest biodegradability (SMY/TMP, %) of 94% as compared to CS
(45%) or CM (47%), indicating that KW is a desirable feedstock for AD. TheIn general, the theoretical results obtained from Buswell equation presented a better agreement with the experimental results.
